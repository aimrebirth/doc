\documentclass[a4paper,12pt]{report}

\usepackage{amsmath, amsthm, amssymb} % formulas
\usepackage{unicode-math}
\usepackage{fontspec}
\setmainfont{Times New Roman}
\usepackage[english,russian]{babel} % for russian hyphenation

\usepackage{xspace}
\usepackage[a4paper,top=25mm,bottom=25mm,left=25mm,right=25mm,includefoot,heightrounded]{geometry} % page geometry

\usepackage{longtable}
\usepackage{array}

\usepackage{tikz}
\usetikzlibrary{shadows}

\usepackage[
	pdftoolbar = false,
	pdffitwindow = true,
	colorlinks = true,
]{hyperref}

% % % % % % % % % % % % % % % % % % % % % % % % % % % % % % % % % % % % % % % %

% paragraph indent
\setlength{\parindent}{1.25cm}

% % % % % % % % % % % % % % % % % % % % % % % % % % % % % % % % % % % % % % % %

% allow russian letters in formulas
\DeclareSymbolFont{cyrletters}{\encodingdefault}{\familydefault}{m}{it}
\newcommand{\makecyrmathletter}[1]{%
  \begingroup\lccode`a=#1\lowercase{\endgroup
  \Umathcode`a}="0 \csname symcyrletters\endcsname\space #1
}
\count255="409
\loop\ifnum\count255<"44F
  \advance\count255 by 1
  \makecyrmathletter{\count255}
\repeat

% % % % % % % % % % % % % % % % % % % % % % % % % % % % % % % % % % % % % % % %

% work settings
\author{lzwdgc}
\title{Polygon-4}
\newcommand{\GraphicsPath}{src/img/}
% work settings

% document settings
%\input{src/pagesetup.tex}

% % % % % % % % % % % % % % % % % % % % % % % % % % % % % % % % % % % % % % % %

\newcommand*\keystroke[1]{%
  \tikz[baseline=(key.base)]
    \node[%
      draw,
      fill=white,
      drop shadow={shadow xshift=0.25ex,shadow yshift=-0.25ex,fill=black,opacity=0.75},
      rectangle,
      rounded corners=2pt,
      inner sep=3pt,
      line width=0.5pt
    ](key) {#1\strut}
  ;
}

% % % % % % % % % % % % % % % % % % % % % % % % % % % % % % % % % % % % % % % %

\newcommand{\ue}{UE4\xspace}
\newcommand{\pf}{Polygon-4\xspace}

% document
\begin{document}
\large % 14pt

\begin{titlepage}
\centering
\vspace*{7cm}
{\huge\bfseries Polygon-4\par}
\vspace{1.5cm}
{\Large Руководство\par}
\vspace{1.5cm}
lzwdgc
\vfill
{\today\par}
\end{titlepage}
\setcounter{page}{2}
\tableofcontents

\chapter*{Введение}
\addcontentsline{toc}{chapter}{Введение}

Графическая составляющая, описываемая в этом документе, относится к игровому движку Unreal Engine 4 (\ue).

Последняя версия данной документации всегда доступна по \href{https://www.dropbox.com/s/ipnakg5civ0xp0g/Polygon4_ru.pdf?dl=1}{ссылке}.

\chapter{Установка}

\section{Пакет установщика}


\textbf{ВАЖНО!} Для корректной работы \pf и любых утилит требуется скачать и установить пакеты Visual C++ Redistributable for Visual Studio 2015 (\href{https://download.microsoft.com/download/9/3/F/93FCF1E7-E6A4-478B-96E7-D4B285925B00/vc_redist.x86.exe}{32 бита}, \href{https://download.microsoft.com/download/9/3/F/93FCF1E7-E6A4-478B-96E7-D4B285925B00/vc_redist.x64.exe}{64 бита}).
Обязательна установка 32 битной версии пакета.
Может потребоваться одновременная установка обеих версий.

Установка \pf производится в полуавтоматическом режиме с использованием программ развёртывания.
Для того, чтобы скачать данную программу, необходимо выполнить следующие действия:

\begin{enumerate}
\item Скачать пакет установщика по \href{https://www.dropbox.com/s/0zhbgb1ftspcv9w/polygon4.zip?dl=1}{ссылке}.
\item Распаковать в папку, в имени которой есть только английские буквы. Например, D:\backslash AIM.
\end{enumerate}

\section{Версия пользователя (релизная)}

Запустить BootstrapRelease.exe для установки релизной версии.
Для игры запустить Polygon4.exe в папке Polygon4Release.

Данная версия обновляется крайне редко и не рекомендуется к использованию на данный момент.

\subsection{Установка}

Каждая установка займёт долгое время (1-2 часа).

\section{Версия nightly (самая новая)}

Запустить BootstrapLatest.exe для установки nightly-версии.
Для игры запустить Polygon4.bat в папке с программой-установщиком.

\subsection{Установка}

Первоначальная установка может занять долгое время (1-2 часа).
Это необходимо из-за специфики процесса распространения установочных файлов проекта.
Последующие обновления проекта производятся запуском той же программы, но, в отличие от первоначальной установки, будут загружены лишь недостающие или изменённые файлы.


\section{Версия разработчика}

\subsection{Зависимости}

Для корректной установки и развёртывания проекта необходимо скачать и установить следующие бесплатные программы.

\begin{enumerate}
\item CMake версии 3.4 или выше (\url{https://cmake.org/}).  \href{https://cmake.org/files/v3.4/cmake-3.4.0-win32-x86.exe}{Скачать}. При установке указать добавление программы в PATH!
\item Git (\url{https://git-scm.com}). \href{https://git-scm.com/download/win}{Скачать}. При установке указать добавление программы в PATH!
\item Visual Studio 2015 Community (\url{https://www.visualstudio.com}). \href{https://go.microsoft.com/fwlink/?LinkId=691978}{Скачать}. При установке обязательно выбрать компоненты Visual C++! Установку производить в директорию по умолчанию на диске C:.
\item Unreal Engine 4 (\url{https://unrealengine.com/}).
\end{enumerate}

\subsection{Установка}

После скачивания и установки программы-установщика, всех зависимостей необходимо запустить BootstrapDeveloper.exe для установки версии разработчика.
Проект будет загружен и распакован в директорию \\Polygon4Developer.
В нём вы найдёте файлы проектов С++ \\(Polygon4Developer/ThirdParty/Engine/Win64/Engine.sln,\\ Polygon4Developer/Polygon4.sln), а также файл проекта \ue\\ (Polygon4Developer/Polygon4.ueproject).

Первоначальная установка может занять долгое время (1-2 часа).
Это необходимо из-за специфики процесса распространения установочных файлов проекта.
Последующие обновления проекта производятся запуском той же программы, но, в отличие от первоначальной установки, будут загружены лишь недостающие или изменённые файлы.


\section{Утилиты}

Запустить BootstrapTools.exe для установки набора утилит.


\section{Обновление установщика}

Иногда, при внесении серьёзных изменений в сам установщик, может потребоваться обновить его.
Для этого необходимо запустить файл BootstrapUpdater.exe.

\section{Журнал работы программ}

Если в работе установщика происходит сбой, то соответствующую информацию об ошибке всегда можно найти в журнале его работы.
Журнал обычно имеет то же имя, что и программа, которая выполнялась, плюс .log.info или .log.trace на конце имени.
Файл, оканчивающийся на .info, является кратким журналом, на .trace - подробным.

\section{Управление}

\begin{longtable}{|c|c|}
\hline

Клавиша & Функция\\ \hline
\endfirsthead

\hline
Клавиша & Функция\\ \hline
\endhead

\keystroke{W} & движение вперёд \\ \hline
\keystroke{A} & движение влево \\ \hline
\keystroke{S} & движение назад \\ \hline
\keystroke{D} & движение вправо \\ \hline
\keystroke{Alt} + \keystroke{Z} & показать/скрыть интерфейс \\ \hline

\keystroke{Pause}, \keystroke{P} & пауза \\ \hline
\keystroke{F1}, \keystroke{V} & сменить вид \\ \hline
\keystroke{Esc} & выход \\ \hline
\keystroke{Shift} & ускорение \\ \hline
\keystroke{Space} & прыжок \\ \hline
ЛКМ & выстрел из лёгкого оружия \\ \hline
ПКМ & выстрел из тяжёлого оружия \\ \hline

\end{longtable}



\chapter{Физика}

\section{Общие положения}

Система координат - трёхмерная, измерения - XYZ. Время изменяется дискретно. Соответственно действия над глайдером производятся в определённые моменты.

В \ue единицей измерения являются юниты - Unreal Units (uu). Будем считать, что
\begin{equation}
1uu = 1см,
\end{equation}
тогда 
\begin{equation}
100uu = 1м.
\end{equation}

Ускорение свободного падения
\begin{equation}
g = 980uu/с^2 = 9.8м/с^2.
\end{equation}

Управление перемещением глайдера осуществляется путём применения к нему различных сил. Движение осуществляется по результирующей силе на каждом тике времени, которая равна сумме всех сил
\begin{equation}
\vec{F} = \vec{F_1} + \vec{F_2} + \dots + \vec{F_n}.
\end{equation}

В любой момент времени на глайдер действует сила тяжести
\begin{equation}
\vec{P} = m\vec{g}.
\end{equation}

\section{Состояние покоя}

В состоянии покоя на глайдер действуют следующие силы.

Сила антиграва
\begin{equation}
\vec{F_a} =
-m\vec{g}.
\end{equation}

\section{Движение вдоль Oz}

Сила антиграва
\begin{equation}
\label{antigrav_force}
\vec{F_a} =
\begin{cases}
0, & \quad \text{при } h_{g} > h_a \\
-m\vec{g}, & \quad \text{при } h_{g} = h_a \\
-m\vec{g}(1+ k_{add}), & \quad \text{при } h_{g} < h_a \\
\end{cases}
\end{equation}
где $h_{g}$ -- высота глайдера над поверхностью, $h_a$ -- уровень, на котором антиграв должен удерживать глайдер над землёй, $k_{add}$ -- коэффициент дополнительной подъёмной силы антиграва, препятствующей столкновению с землёй.
Сила антиграва направлена строго вверх.

При $h_g < h_a$ сила отталкивания тем больше, чем меньше расстояние до земли
\begin{equation}
k_{add} =\frac{h_{g}}{h_{a}}.
\end{equation}

Для обеспечения плавности снижения допустимо ввести силу антиграва при $h_{g} > h_a$. Таким образом, формула \eqref{antigrav_force} будет иметь вид

\begin{equation}
\vec{F_a} =
\begin{cases}
0, & \quad \text{при } h_{g} > k h_a \\
-m\vec{g}(1+ k_{add}), & \quad \text{при } h_{g} \le k h_a \\
\end{cases}
\end{equation}
где $k=2,3,\dots$. Примем $k=3$.

\chapter{Оборудование и глайдеры}
\section{Глайдеры}
\section{Оружие}
\section{Оборудование}

Антигравитатор (антиграв) --- устройство, позволяющее глайдеру "<парить"> над землёй.

\chapter{Программный код}

Исходные коды проекта, утилит и документации находятся в открытом доступе на GitHub, и их можно найти по \href{https://github.com/aimrebirth}{ссылке}.

\section{Спецификаторы строк}

\begin{longtable}{|l|l|}
\hline

Спецификатор & Описание\\ \hline
\endfirsthead

\hline
Спецификатор & Описание\\ \hline
\endhead

\%bn & название здания \\ \hline
\%pn & имя игрока \\ \hline
\%rn & название текущего уровня рейтинга \\ \hline

\end{longtable}

\chapter*{Заключение}
\addcontentsline{toc}{chapter}{Заключение}

\chapter*{Полезные ссылки}
\addcontentsline{toc}{chapter}{Полезные ссылки}

\begin{itemize}
\item \href{https://github.com/aimrebirth}{Проект на GitHub}
\item \href{https://www.dropbox.com/s/0zhbgb1ftspcv9w/polygon4.zip?dl=1}{Пакет установщика}
\item \href{https://www.dropbox.com/s/ipnakg5civ0xp0g/Polygon4_ru.pdf?dl=1}{Документация}
\item \href{http://aim-fans.ru/index.php?showtopic=147}{Тема на aim-fans.ru}
\item \href{http://forum.skyriver.ru/viewtopic.php?f=26&t=4801}{Тема на skyriver.ru}
\end{itemize}

\end{document}
